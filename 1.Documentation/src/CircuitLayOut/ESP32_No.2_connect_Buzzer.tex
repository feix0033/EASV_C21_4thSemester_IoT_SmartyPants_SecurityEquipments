%! Author = Evens
%! Date = 16/03/2023

% Preamble
\documentclass[11pt]{article}

% Packages
\usepackage{amsmath}
\usepackage{circuitikz}

% Document
\begin{document}
    \begin{circuitikz}

        \ctikzset{multipoles/dipchip/width = 2}
        \ctikzset{multipoles/external pins thickness=2}

        \draw (0,0)
        node [dipchip, num pins = 12, no topmark, hide numbers, draw only pins={12,11,10}, external pins width=0.2,
        external pad fraction=6](E){ESP32 No.2};

        \node [left, font=\tiny] at (E.bpin 12) {$3.3V$};
        \node [left, font=\tiny] at (E.bpin 11) {$12$};
        \node [left, font=\tiny] at (E.bpin 10) {$GUN$};


        \draw (6,1)
        node [dipchip, num pins = 6, no topmark, hide numbers, external pins width=0.2,
        external pad fraction=6](B){buzzer};

        \node [right, font=\tiny] at (B.bpin 1) {$3.3V$};
        \node [right, font=\tiny] at (B.bpin 2) {$SIG$};
        \node [right, font=\tiny] at (B.bpin 3) {$GUN$};

        \draw (E.pin 12)[thick] -- ++(1,0)[color=red] to (B.pin 1);
        \draw (E.pin 11)[thick] -- ++(1,0)[color=yellow] to (B.pin 2);
        \draw (E.pin 10)[thick] -- ++(1,0)[color=black] to (B.pin 3);


    \end{circuitikz}

\end{document}