%! Author = Evens
%! Date = 20/02/2023

% Preamble
\documentclass[11pt]{article}

% Packages
% import the package circuitikz, [siunitx] is working for add label to get more info
\usepackage{circuitikz}

% Document
\begin{document}
    % set the circuiTikZ env
    % use [scale=2] to zoom the size as 2 times
    \begin{circuitikz}
        % draw the element from position (0,0)
        \draw(0,0)

        % notice:
        % to[component]
        % (the position where the component going by x,y)
        % -- (the end point of component to next, using the line to connect by x,y)

        % draw a battery from (0,0) to (0,4)
        % that means go up to 4 unit draw
        to[battery] (0,4)

        % draw an ammeter from (4,0) to (4,4)
        % use parameter 'l=' to add a label
        % '<\ampere>' means ampere as unit
        % '\milli' means milli
        % '_' after 'l' means under the simble make the label
        to[ammeter, l_=2<\milli\ampere>] (4,4)

        % add a Capacitor between ammeter and lamp
        % make a label next to the simble with an arrow useing 'i='
        % 'o-o' using for add empty dot two side of capacitor
        to[C, o-o, l=3<\farad>, i= capacitor] (4,0)
        --( 3.5,0)

        % draw an lamp from (3.5,0) to (0.5,0)
        % '*-*' using for add dot at to side of lamp
        % from x=0.5 go down 2cm
        to[lamp, *-*](0.5,0)

        % make a line to close the circle back to (0,0)
        -- (0, 0)

        % make a line go down from (0.5,0) to ( 0.5, -2)
        (0.5,0) -- (0.5,-2)

        % draw a voltmeter
        % change the component as red by 'color='
        to[voltmeter, l=3<\kilo\volt>, color=red]
        % from x=3.5 go up back to 0
        % auto it will draw a voltmeter from 0.5 to 3.5
        (3.5,-2) -- (3.5,0)


        % add some single componet use '(x,y) node[component] {}'
        % rember the (x,y) is base the last component position,
        % which means (0,-2) are from the voltmeter (0.5,-2) go to (0,-3)
        (-1,-3) node[transformer] {}

        % remember close with a semicolon
        ;


    \end{circuitikz}
\end{document}
