%! Author = Evens
%! Date = 16/03/2023

% Preamble
\documentclass[11pt]{article}

% Packages
\usepackage{amsmath}
\usepackage{circuitikz}


% Document
\begin{document}
    \begin{circuitikz}
    \ctikzset{multipoles/dipchip/width = 2}
    \ctikzset{multipoles/external pins thickness=2}

    \draw (0,0)
    node [dipchip, num pins = 8, no topmark, hide numbers, draw only pins={5-8}, external pins width=0.2,
    external pad fraction=6](C){
    TypeC Adapter};
    \node [left, font=\tiny] at (C.bpin 8) {$VCC$};
    \node [left, font=\tiny] at (C.bpin 7) {$RX0$};
    \node [left, font=\tiny] at (C.bpin 6) {$TX0$};
    \node [left, font=\tiny] at (C.bpin 5) {$GUN$};


    \draw (8,0)
    node [dipchip, num pins = 12, no topmark, hide numbers,draw only pins={1,3,4,6,10,9}, external pins width=0.2,
    external pad
    fraction=6](CAM){
    ESP32CAM};
    \node [right, font=\tiny] at (CAM.bpin 1) {$3.3V$};
    \node [right, font=\tiny] at (CAM.bpin 3) {$RX0$};
    \node [right, font=\tiny] at (CAM.bpin 4) {$TX0$};
    \node [right, font=\tiny] at (CAM.bpin 6) {$GUN$};
    \node [left, font=\tiny] at (CAM.bpin 10) {$GPIO0$};
    \node [left, font=\tiny] at (CAM.bpin 9) {$GUN$};


    \draw (C.pin 8) -- ++(4,0) to(CAM.pin 1);
    \draw (C.pin 7) -- ++(4,0) to(CAM.pin 3);
    \draw (C.pin 6) -- ++(4,0) to(CAM.pin 4);
    \draw (C.pin 5) -- ++(4,0) to(CAM.pin 6);
    \draw (CAM.pin 10) -- ++(2,0) -- ++(0,-0.5) -- ++(-2,0) to (CAM.pin 9);

    \end{circuitikz}
\end{document}